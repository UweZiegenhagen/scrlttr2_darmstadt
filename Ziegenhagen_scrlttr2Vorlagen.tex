\documentclass[12pt,ngerman]{beamer}


\usepackage[utf8]{inputenc}
\usepackage[T1]{fontenc}
\usepackage{booktabs}
\usepackage{babel}
\usepackage{graphicx}
\usepackage{csquotes}
\usepackage{xcolor}



\usepackage[]{listings}

%%%%%%%
\definecolor{hellgelb}{rgb}{1,1,0.8}
\definecolor{lightgelb}{rgb}{1,1,0.8}
\definecolor{colKeys}{rgb}{0,0,1}
\definecolor{colIdentifier}{rgb}{0,0,0}
\definecolor{colComments}{rgb}{1,0,0}
\definecolor{colString}{rgb}{0,0.5,0}


\lstset{%
		language={[LaTeX]TeX},%
		morekeywords={opening, closing, setkomavar, getkomavar},%
    float=hbp,%
    basicstyle=\ttfamily\footnotesize, %
    identifierstyle=\color{colIdentifier}, %
    keywordstyle=\color{colKeys}, %
    stringstyle=\color{colString}, %
    commentstyle=\color{colComments}, %
    literate={fl}{{f{}l}}2,%
    columns=flexible, %
    tabsize=2, %
    frame=single, %
    upquote=true,%
    extendedchars=true, %
    showspaces=false, %
    showstringspaces=false, %
    numbers=left, %
    numberstyle=\tiny, %
    breaklines=true, %
    backgroundcolor=\color{hellgelb}, %
    breakautoindent=true, %
    captionpos=b%
}




%%%%%%%%%%%%
\lstset{literate=%
    {Ö}{{\"O}}1
    {Ä}{{\"A}}1
    {Ü}{{\"U}}1
    {ß}{{\ss}}1
    {ü}{{\"u}}1
    {ä}{{\"a}}1
    {ö}{{\"o}}1
    {~}{{\textasciitilde}}1
}


\author{Uwe Ziegenhagen}
\title{Briefvorlagen mit \texttt{scrlttr2}}

\begin{document}


\begin{frame}

\maketitle

\end{frame}

\begin{frame}
\frametitle{Einführung}

\begin{itemize}
\item \enquote{ Ich finde, dass man Latex-Dokumenten meistens ansieht, dass sie in Latex gesetzt wurden, das stört mich, wenn ich auch die Vorteile erkenne.} und der Verweis auf \texttt{dinbrief}
\item \LaTeX-Layout muss nicht (schlecht) sein!
\item Für Briefe nutze ich \texttt{scrlttr2}, sehr flexibles Paket für Briefe
\item erlaubt (leichte) Anpassung an Design-Vorgaben
\item Ziel des Vortrags

\begin{itemize}
	\item Kurze Einführung in \texttt{scrlttr2}
	\item Erstellung von Briefvorlagen
\end{itemize}


\end{itemize}
\end{frame}

\begin{frame}[fragile]
\frametitle{Ein minimaler Brief}

\lstinputlisting{brief-01.tex}

\end{frame}

\begin{frame}[plain]
\frametitle{Ergebnis}

\begin{center}
\fbox{\includegraphics[trim=0cm 14cm 0cm 3cm, width=1\textwidth]{brief-01}}
\end{center}
\end{frame}


\begin{frame}
\frametitle{Vordefinierte Variablen}

\texttt{scrlttr2} kennt eine Vielzahl vordefinierter Variablen, die sich befüllen lassen. Hier die wichtigsten: \vspace*{1em}

\begin{columns}
\begin{column}{0.3\textwidth}
\begin{itemize}
	\item backaddress
	\item customer
	\item date
	\item firstfoot
	\item firsthead
	\item fromaddress
	\item frombank
	\item fromemail
	\end{itemize}
\end{column}
\begin{column}{0.3\textwidth}
\begin{itemize}
	\item fromname
	\item fromfax
	\item fromphone
	\item fromurl
	\item invoice
	\item location
	\item myref
	\item nextfoot
	\end{itemize}
\end{column}
\begin{column}{0.3\textwidth}
\begin{itemize}
	\item nexthead
	\item place
	\item signature
	\item subject
	\item toname
	\item toaddress
	\item yourmail
	\item yourref
	\end{itemize}
\end{column}

\end{columns}

\end{frame}

\begin{frame}[containsverbatim]
\frametitle{Setzen und Nutzen von Variablen}

\begin{itemize}
	\item \lstinline|\setkomavar{Variable}{Wert}| weist der Variablen einen Wert zu
	\item \lstinline|\setkomavar{Variable}[Label]{Wert}| weist der Variablen einen Wert zu und setzt das Label
	\item \lstinline|\setkomavar*{Variable}{Label}| ändert nur das Label
\end{itemize}
\end{frame}

\begin{frame}[fragile]
\frametitle{Setzen des Absenders}

\lstinputlisting{brief-02.tex}

\end{frame}

\begin{frame}[plain]
\frametitle{Ergebnis}

\begin{center}
\fbox{\includegraphics[trim=0cm 15cm 0cm 0cm, width=1\textwidth]{brief-02}}
\end{center}
\end{frame}

\begin{frame}[fragile]
\frametitle{Setzen des Absenders}

\lstinputlisting{brief-03.tex}

\end{frame}

\begin{frame}[plain]
\frametitle{Ergebnis}

\begin{center}
\fbox{\includegraphics[trim=0cm 15cm 0cm 0cm, width=1\textwidth]{brief-03}}
\end{center}
\end{frame}



\end{document}