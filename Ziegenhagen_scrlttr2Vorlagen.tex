\documentclass[12pt,ngerman]{beamer}


\usepackage[utf8]{inputenc}
\usepackage[T1]{fontenc}
\usepackage{booktabs}
\usepackage{babel}
\usepackage{graphicx}
\usepackage{csquotes}
\usepackage{xcolor}



\usepackage[]{listings}

%%%%%%%
\definecolor{hellgelb}{rgb}{1,1,0.8}
\definecolor{lightgelb}{rgb}{1,1,0.8}
\definecolor{colKeys}{rgb}{0,0,1}
\definecolor{colIdentifier}{rgb}{0,0,0}
\definecolor{colComments}{rgb}{1,0,0}
\definecolor{colString}{rgb}{0,0.5,0}


\lstset{%
    float=hbp,%
    basicstyle=\ttfamily\footnotesize, %
    identifierstyle=\color{colIdentifier}, %
    keywordstyle=\color{colKeys}, %
    stringstyle=\color{colString}, %
    commentstyle=\color{colComments}, %
    literate={fl}{{f{}l}}2,%
    columns=flexible, %
    tabsize=2, %
    frame=single, %
    upquote=true,%
    extendedchars=true, %
    showspaces=false, %
    showstringspaces=false, %
    numbers=left, %
    numberstyle=\tiny, %
    breaklines=true, %
    backgroundcolor=\color{hellgelb}, %
    breakautoindent=true, %
    captionpos=b%
}




%%%%%%%%%%%%
\lstset{literate=%
    {Ö}{{\"O}}1
    {Ä}{{\"A}}1
    {Ü}{{\"U}}1
    {ß}{{\ss}}1
    {ü}{{\"u}}1
    {ä}{{\"a}}1
    {ö}{{\"o}}1
    {~}{{\textasciitilde}}1
}


\author{Uwe Ziegenhagen}
\title{Briefvorlagen mit \texttt{scrlttr2}}

\begin{document}


\begin{frame}

\maketitle

\end{frame}

\begin{frame}
\frametitle{Einführung}

\begin{itemize}
\item \enquote{ Ich finde, dass man Latex-Dokumenten meistens ansieht, dass sie in Latex gesetzt wurden, das stört mich, wenn ich auch die Vorteile erkenne.} und der Verweis auf \texttt{dinbrief}
\item \LaTeX-Layout muss nicht (schlecht) sein!
\item Für Briefe nutze ich \texttt{scrlttr2}, sehr flexibles Paket für Briefe
\item erlaubt (leichte) Anpassung an Design-Vorgaben
\item Ziel des Vortrags: komplexes Design umsetzen
\end{itemize}
\end{frame}

\begin{frame}[fragile]
\frametitle{}

\lstinputlisting[language={[LaTeX]TeX},morekeywords={opening, closing}]{brief-01.tex}

\end{frame}


\end{document}