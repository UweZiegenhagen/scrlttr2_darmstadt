\documentclass[12pt,ngerman]{beamer}


\usepackage[utf8]{inputenc}
\usepackage[T1]{fontenc}
\usepackage{booktabs}
\usepackage{babel}
\usepackage{graphicx}
\usepackage{csquotes}
\usepackage{xcolor}

\author{Uwe Ziegenhagen}
\title{Briefvorlagen mit \texttt{scrlttr2}}

\begin{document}


\begin{frame}

\maketitle

\end{frame}

\begin{frame}
\frametitle{Einführung}

\begin{itemize}
\item \enquote{ Ich finde, dass man Latex-Dokumenten meistens ansieht, dass sie in Latex gesetzt wurden, das stört mich, wenn ich auch die Vorteile erkenne.} und der Verweis auf \texttt{dinbrief}
\item \LaTeX-Layout muss nicht (schlecht) sein!
\item Für Briefe nutze ich \texttt{scrlttr2}, sehr flexibles Paket für Briefe
\item erlaubt (leichte) Anpassung an Design-Vorgaben
\item Ziel des Vortrags: komplexes Design umsetzen
\end{itemize}
\end{frame}


\end{document}